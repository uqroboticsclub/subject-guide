
\courseTemplate[
code = {COMS4105},
title = {Communication Systems},
score = 5,
prereq = {CSSE3010 + ELEC3004},
contact = {2L, 2T, 2P},
coordinator = {Dr Konstanty Bialkowski (ksb@itee.uq.edu.au)},
assessment = {
Introductory Assignment & 10\% & A simple assignment due early in the semester \\
Practical Milestones & 15\% & Difficult practical problems assessed every 2 weeks during the semester, normally require the use of Matlab or Python to decode a signal\\
Project Report & 15\% & A report due at the end of semester on a random topic \\
Final Exam & 45\% & Final exam\\
},
review = {
    COMS4105 is a great course, and if you are interested in radio communications at all you should take it. The practical milestones are extremely fun, however are time consuming. Don't expect to get them all completed, they are pretty massive and full marks are normally awarded to reasonable attempts (not that you'll ever find out what you got). The final report is also interesting and you beneift from starting early as it normally helps to do the last practical assignment.
},
preparation = {
    \item Get good at MATLAB, Python and C++ are also options but MATLAB has all the librarys built in (less work for you).
    \item Look up GNURadio
    \item Revise moduleation schemes from CSSE3010 (Hamming, CRC, Virterbi, etc)
}]{}
